


% Header, overrides base

    % Make sure that the sphinx doc style knows who it inherits from.
    \def\sphinxdocclass{article}

    % Declare the document class
    \documentclass[letterpaper,10pt,english]{/Users/juan/Envs/pythonslides/lib/python3.3/site-packages/sphinx/texinputs/sphinxhowto}

    % Imports
    \usepackage[utf8]{inputenc}
    \DeclareUnicodeCharacter{00A0}{\\nobreakspace}
    \usepackage[T1]{fontenc}
    \usepackage{babel}
    \usepackage{times}
    \usepackage{import}
    \usepackage[Bjarne]{/Users/juan/Envs/pythonslides/lib/python3.3/site-packages/sphinx/texinputs/fncychap}
    \usepackage{longtable}
    \usepackage{/Users/juan/Envs/pythonslides/lib/python3.3/site-packages/sphinx/texinputs/sphinx}
    \usepackage{multirow}

    \usepackage{amsmath}
    \usepackage{amssymb}
    \usepackage{ucs}
    \usepackage{enumerate}

    % Used to make the Input/Output rules follow around the contents.
    \usepackage{needspace}

    % Pygments requirements
    \usepackage{fancyvrb}
    \usepackage{color}
    % ansi colors additions
    \definecolor{darkgreen}{rgb}{.12,.54,.11}
    \definecolor{lightgray}{gray}{.95}
    \definecolor{brown}{rgb}{0.54,0.27,0.07}
    \definecolor{purple}{rgb}{0.5,0.0,0.5}
    \definecolor{darkgray}{gray}{0.25}
    \definecolor{lightred}{rgb}{1.0,0.39,0.28}
    \definecolor{lightgreen}{rgb}{0.48,0.99,0.0}
    \definecolor{lightblue}{rgb}{0.53,0.81,0.92}
    \definecolor{lightpurple}{rgb}{0.87,0.63,0.87}
    \definecolor{lightcyan}{rgb}{0.5,1.0,0.83}

    % Needed to box output/input
    \usepackage{tikz}
        \usetikzlibrary{calc,arrows,shadows}
    \usepackage[framemethod=tikz]{mdframed}

    \usepackage{alltt}

    % Used to load and display graphics
    \usepackage{graphicx}
    \graphicspath{ {figs/} }
    \usepackage[Export]{adjustbox} % To resize

    % used so that images for notebooks which have spaces in the name can still be included
    \usepackage{grffile}


    % For formatting output while also word wrapping.
    \usepackage{listings}
    \lstset{breaklines=true}
    \lstset{basicstyle=\small\ttfamily}
    \def\smaller{\fontsize{9.5pt}{9.5pt}\selectfont}

    %Pygments definitions
    
\makeatletter
\def\PY@reset{\let\PY@it=\relax \let\PY@bf=\relax%
    \let\PY@ul=\relax \let\PY@tc=\relax%
    \let\PY@bc=\relax \let\PY@ff=\relax}
\def\PY@tok#1{\csname PY@tok@#1\endcsname}
\def\PY@toks#1+{\ifx\relax#1\empty\else%
    \PY@tok{#1}\expandafter\PY@toks\fi}
\def\PY@do#1{\PY@bc{\PY@tc{\PY@ul{%
    \PY@it{\PY@bf{\PY@ff{#1}}}}}}}
\def\PY#1#2{\PY@reset\PY@toks#1+\relax+\PY@do{#2}}

\expandafter\def\csname PY@tok@m\endcsname{\def\PY@tc##1{\textcolor[rgb]{0.40,0.40,0.40}{##1}}}
\expandafter\def\csname PY@tok@o\endcsname{\def\PY@tc##1{\textcolor[rgb]{0.40,0.40,0.40}{##1}}}
\expandafter\def\csname PY@tok@k\endcsname{\let\PY@bf=\textbf\def\PY@tc##1{\textcolor[rgb]{0.00,0.50,0.00}{##1}}}
\expandafter\def\csname PY@tok@c\endcsname{\let\PY@it=\textit\def\PY@tc##1{\textcolor[rgb]{0.25,0.50,0.50}{##1}}}
\expandafter\def\csname PY@tok@ne\endcsname{\let\PY@bf=\textbf\def\PY@tc##1{\textcolor[rgb]{0.82,0.25,0.23}{##1}}}
\expandafter\def\csname PY@tok@s2\endcsname{\def\PY@tc##1{\textcolor[rgb]{0.73,0.13,0.13}{##1}}}
\expandafter\def\csname PY@tok@w\endcsname{\def\PY@tc##1{\textcolor[rgb]{0.73,0.73,0.73}{##1}}}
\expandafter\def\csname PY@tok@sc\endcsname{\def\PY@tc##1{\textcolor[rgb]{0.73,0.13,0.13}{##1}}}
\expandafter\def\csname PY@tok@c1\endcsname{\let\PY@it=\textit\def\PY@tc##1{\textcolor[rgb]{0.25,0.50,0.50}{##1}}}
\expandafter\def\csname PY@tok@bp\endcsname{\def\PY@tc##1{\textcolor[rgb]{0.00,0.50,0.00}{##1}}}
\expandafter\def\csname PY@tok@mf\endcsname{\def\PY@tc##1{\textcolor[rgb]{0.40,0.40,0.40}{##1}}}
\expandafter\def\csname PY@tok@vc\endcsname{\def\PY@tc##1{\textcolor[rgb]{0.10,0.09,0.49}{##1}}}
\expandafter\def\csname PY@tok@kp\endcsname{\def\PY@tc##1{\textcolor[rgb]{0.00,0.50,0.00}{##1}}}
\expandafter\def\csname PY@tok@kr\endcsname{\let\PY@bf=\textbf\def\PY@tc##1{\textcolor[rgb]{0.00,0.50,0.00}{##1}}}
\expandafter\def\csname PY@tok@kt\endcsname{\def\PY@tc##1{\textcolor[rgb]{0.69,0.00,0.25}{##1}}}
\expandafter\def\csname PY@tok@mo\endcsname{\def\PY@tc##1{\textcolor[rgb]{0.40,0.40,0.40}{##1}}}
\expandafter\def\csname PY@tok@s1\endcsname{\def\PY@tc##1{\textcolor[rgb]{0.73,0.13,0.13}{##1}}}
\expandafter\def\csname PY@tok@mh\endcsname{\def\PY@tc##1{\textcolor[rgb]{0.40,0.40,0.40}{##1}}}
\expandafter\def\csname PY@tok@mi\endcsname{\def\PY@tc##1{\textcolor[rgb]{0.40,0.40,0.40}{##1}}}
\expandafter\def\csname PY@tok@kd\endcsname{\let\PY@bf=\textbf\def\PY@tc##1{\textcolor[rgb]{0.00,0.50,0.00}{##1}}}
\expandafter\def\csname PY@tok@nd\endcsname{\def\PY@tc##1{\textcolor[rgb]{0.67,0.13,1.00}{##1}}}
\expandafter\def\csname PY@tok@vg\endcsname{\def\PY@tc##1{\textcolor[rgb]{0.10,0.09,0.49}{##1}}}
\expandafter\def\csname PY@tok@kc\endcsname{\let\PY@bf=\textbf\def\PY@tc##1{\textcolor[rgb]{0.00,0.50,0.00}{##1}}}
\expandafter\def\csname PY@tok@kn\endcsname{\let\PY@bf=\textbf\def\PY@tc##1{\textcolor[rgb]{0.00,0.50,0.00}{##1}}}
\expandafter\def\csname PY@tok@s\endcsname{\def\PY@tc##1{\textcolor[rgb]{0.73,0.13,0.13}{##1}}}
\expandafter\def\csname PY@tok@il\endcsname{\def\PY@tc##1{\textcolor[rgb]{0.40,0.40,0.40}{##1}}}
\expandafter\def\csname PY@tok@ow\endcsname{\let\PY@bf=\textbf\def\PY@tc##1{\textcolor[rgb]{0.67,0.13,1.00}{##1}}}
\expandafter\def\csname PY@tok@cm\endcsname{\let\PY@it=\textit\def\PY@tc##1{\textcolor[rgb]{0.25,0.50,0.50}{##1}}}
\expandafter\def\csname PY@tok@gd\endcsname{\def\PY@tc##1{\textcolor[rgb]{0.63,0.00,0.00}{##1}}}
\expandafter\def\csname PY@tok@ge\endcsname{\let\PY@it=\textit}
\expandafter\def\csname PY@tok@gh\endcsname{\let\PY@bf=\textbf\def\PY@tc##1{\textcolor[rgb]{0.00,0.00,0.50}{##1}}}
\expandafter\def\csname PY@tok@gi\endcsname{\def\PY@tc##1{\textcolor[rgb]{0.00,0.63,0.00}{##1}}}
\expandafter\def\csname PY@tok@go\endcsname{\def\PY@tc##1{\textcolor[rgb]{0.53,0.53,0.53}{##1}}}
\expandafter\def\csname PY@tok@gp\endcsname{\let\PY@bf=\textbf\def\PY@tc##1{\textcolor[rgb]{0.00,0.00,0.50}{##1}}}
\expandafter\def\csname PY@tok@gr\endcsname{\def\PY@tc##1{\textcolor[rgb]{1.00,0.00,0.00}{##1}}}
\expandafter\def\csname PY@tok@gs\endcsname{\let\PY@bf=\textbf}
\expandafter\def\csname PY@tok@gt\endcsname{\def\PY@tc##1{\textcolor[rgb]{0.00,0.27,0.87}{##1}}}
\expandafter\def\csname PY@tok@gu\endcsname{\let\PY@bf=\textbf\def\PY@tc##1{\textcolor[rgb]{0.50,0.00,0.50}{##1}}}
\expandafter\def\csname PY@tok@no\endcsname{\def\PY@tc##1{\textcolor[rgb]{0.53,0.00,0.00}{##1}}}
\expandafter\def\csname PY@tok@cp\endcsname{\def\PY@tc##1{\textcolor[rgb]{0.74,0.48,0.00}{##1}}}
\expandafter\def\csname PY@tok@vi\endcsname{\def\PY@tc##1{\textcolor[rgb]{0.10,0.09,0.49}{##1}}}
\expandafter\def\csname PY@tok@cs\endcsname{\let\PY@it=\textit\def\PY@tc##1{\textcolor[rgb]{0.25,0.50,0.50}{##1}}}
\expandafter\def\csname PY@tok@ni\endcsname{\let\PY@bf=\textbf\def\PY@tc##1{\textcolor[rgb]{0.60,0.60,0.60}{##1}}}
\expandafter\def\csname PY@tok@sx\endcsname{\def\PY@tc##1{\textcolor[rgb]{0.00,0.50,0.00}{##1}}}
\expandafter\def\csname PY@tok@nn\endcsname{\let\PY@bf=\textbf\def\PY@tc##1{\textcolor[rgb]{0.00,0.00,1.00}{##1}}}
\expandafter\def\csname PY@tok@nl\endcsname{\def\PY@tc##1{\textcolor[rgb]{0.63,0.63,0.00}{##1}}}
\expandafter\def\csname PY@tok@nc\endcsname{\let\PY@bf=\textbf\def\PY@tc##1{\textcolor[rgb]{0.00,0.00,1.00}{##1}}}
\expandafter\def\csname PY@tok@nb\endcsname{\def\PY@tc##1{\textcolor[rgb]{0.00,0.50,0.00}{##1}}}
\expandafter\def\csname PY@tok@na\endcsname{\def\PY@tc##1{\textcolor[rgb]{0.49,0.56,0.16}{##1}}}
\expandafter\def\csname PY@tok@nf\endcsname{\def\PY@tc##1{\textcolor[rgb]{0.00,0.00,1.00}{##1}}}
\expandafter\def\csname PY@tok@sr\endcsname{\def\PY@tc##1{\textcolor[rgb]{0.73,0.40,0.53}{##1}}}
\expandafter\def\csname PY@tok@ss\endcsname{\def\PY@tc##1{\textcolor[rgb]{0.10,0.09,0.49}{##1}}}
\expandafter\def\csname PY@tok@sh\endcsname{\def\PY@tc##1{\textcolor[rgb]{0.73,0.13,0.13}{##1}}}
\expandafter\def\csname PY@tok@si\endcsname{\let\PY@bf=\textbf\def\PY@tc##1{\textcolor[rgb]{0.73,0.40,0.53}{##1}}}
\expandafter\def\csname PY@tok@sd\endcsname{\let\PY@it=\textit\def\PY@tc##1{\textcolor[rgb]{0.73,0.13,0.13}{##1}}}
\expandafter\def\csname PY@tok@se\endcsname{\let\PY@bf=\textbf\def\PY@tc##1{\textcolor[rgb]{0.73,0.40,0.13}{##1}}}
\expandafter\def\csname PY@tok@nt\endcsname{\let\PY@bf=\textbf\def\PY@tc##1{\textcolor[rgb]{0.00,0.50,0.00}{##1}}}
\expandafter\def\csname PY@tok@nv\endcsname{\def\PY@tc##1{\textcolor[rgb]{0.10,0.09,0.49}{##1}}}
\expandafter\def\csname PY@tok@sb\endcsname{\def\PY@tc##1{\textcolor[rgb]{0.73,0.13,0.13}{##1}}}
\expandafter\def\csname PY@tok@err\endcsname{\def\PY@bc##1{\setlength{\fboxsep}{0pt}\fcolorbox[rgb]{1.00,0.00,0.00}{1,1,1}{\strut ##1}}}

\def\PYZbs{\char`\\}
\def\PYZus{\char`\_}
\def\PYZob{\char`\{}
\def\PYZcb{\char`\}}
\def\PYZca{\char`\^}
\def\PYZam{\char`\&}
\def\PYZlt{\char`\<}
\def\PYZgt{\char`\>}
\def\PYZsh{\char`\#}
\def\PYZpc{\char`\%}
\def\PYZdl{\char`\$}
\def\PYZhy{\char`\-}
\def\PYZsq{\char`\'}
\def\PYZdq{\char`\"}
\def\PYZti{\char`\~}
% for compatibility with earlier versions
\def\PYZat{@}
\def\PYZlb{[}
\def\PYZrb{]}
\makeatother


    %Set pygments styles if needed...
    
        \definecolor{nbframe-border}{rgb}{0.867,0.867,0.867}
        \definecolor{nbframe-bg}{rgb}{0.969,0.969,0.969}
        \definecolor{nbframe-in-prompt}{rgb}{0.0,0.0,0.502}
        \definecolor{nbframe-out-prompt}{rgb}{0.545,0.0,0.0}

        \newenvironment{ColorVerbatim}
        {\begin{mdframed}[%
            roundcorner=1.0pt, %
            backgroundcolor=nbframe-bg, %
            userdefinedwidth=1\linewidth, %
            leftmargin=0.1\linewidth, %
            innerleftmargin=0pt, %
            innerrightmargin=0pt, %
            linecolor=nbframe-border, %
            linewidth=1pt, %
            usetwoside=false, %
            everyline=true, %
            innerlinewidth=3pt, %
            innerlinecolor=nbframe-bg, %
            middlelinewidth=1pt, %
            middlelinecolor=nbframe-bg, %
            outerlinewidth=0.5pt, %
            outerlinecolor=nbframe-border, %
            needspace=0pt
        ]}
        {\end{mdframed}}
        
        \newenvironment{InvisibleVerbatim}
        {\begin{mdframed}[leftmargin=0.1\linewidth,innerleftmargin=3pt,innerrightmargin=3pt, userdefinedwidth=1\linewidth, linewidth=0pt, linecolor=white, usetwoside=false]}
        {\end{mdframed}}

        \renewenvironment{Verbatim}[1][\unskip]
        {\begin{alltt}\smaller}
        {\end{alltt}}
    

    % Help prevent overflowing lines due to urls and other hard-to-break 
    % entities.  This doesn't catch everything...
    \sloppy

    % Document level variables
    \title{Notas}
    \date{January 20, 2014}
    \release{}
    \author{Unknown Author}
    \renewcommand{\releasename}{}

    % TODO: Add option for the user to specify a logo for his/her export.
    \newcommand{\sphinxlogo}{}

    % Make the index page of the document.
    \makeindex

    % Import sphinx document type specifics.
     


% Body

    % Start of the document
    \begin{document}

        
            \maketitle
        

        


        
        \section{Primeros pasos Python}\label{primeros-pasos-python}Al igual que otros lenguajes de programación Python tiene los tipos
usuales de datos: booleanos (bool), enteros (int), flotantes (float) y
palabras (strings). Es importante decir que cada uno está definido como
un objeto. Esto qué quiere decir, lo que nos quiere decir es que cada
uno de ellos tiene definidas ciertas propiedades y métodos para trabajar
con ellas.\subsection{Palabras}\label{palabras}

Podemos definir una palabra y asignarla a una variable de la siguiente
manera, el uso de comilla simple puede ser sutituido por comillas dobles
sin ningún problema,

    % Make sure that atleast 4 lines are below the HR
    \needspace{4\baselineskip}

    
        \vspace{6pt}
        \makebox[0.1\linewidth]{\smaller\hfill\tt\color{nbframe-in-prompt}In\hspace{4pt}{[}1{]}:\hspace{4pt}}\\*
        \vspace{-2.65\baselineskip}
        \begin{ColorVerbatim}
            \vspace{-0.7\baselineskip}
            \begin{Verbatim}[commandchars=\\\{\}]
\PY{n}{nombre} \PY{o}{=} \PY{l+s}{\PYZsq{}}\PY{l+s}{Catalina}\PY{l+s}{\PYZsq{}}
\end{Verbatim}

            
                \vspace{-0.2\baselineskip}
            
        \end{ColorVerbatim}
    


    % Make sure that atleast 4 lines are below the HR
    \needspace{4\baselineskip}

    
        \vspace{6pt}
        \makebox[0.1\linewidth]{\smaller\hfill\tt\color{nbframe-in-prompt}In\hspace{4pt}{[}2{]}:\hspace{4pt}}\\*
        \vspace{-2.65\baselineskip}
        \begin{ColorVerbatim}
            \vspace{-0.7\baselineskip}
            \begin{Verbatim}[commandchars=\\\{\}]
\PY{n}{apellido} \PY{o}{=} \PY{l+s}{\PYZdq{}}\PY{l+s}{Valderrama}\PY{l+s}{\PYZdq{}}
\end{Verbatim}

            
                \vspace{-0.2\baselineskip}
            
        \end{ColorVerbatim}
    
Para conocer los métodos que pueden ser aplicados en un objeto podemos
listarlos así

    % Make sure that atleast 4 lines are below the HR
    \needspace{4\baselineskip}

    
        \vspace{6pt}
        \makebox[0.1\linewidth]{\smaller\hfill\tt\color{nbframe-in-prompt}In\hspace{4pt}{[}3{]}:\hspace{4pt}}\\*
        \vspace{-2.65\baselineskip}
        \begin{ColorVerbatim}
            \vspace{-0.7\baselineskip}
            \begin{Verbatim}[commandchars=\\\{\}]
\PY{n+nb}{dir}\PY{p}{(}\PY{n+nb}{str}\PY{p}{)}
\end{Verbatim}

            
                \vspace{-0.2\baselineskip}
            
        \end{ColorVerbatim}
    

    

        % If the first block is an image, minipage the image.  Else
        % request a certain amount of space for the input text.
        \needspace{4\baselineskip}
        
        

            % Add document contents.
            
                \makebox[0.1\linewidth]{\smaller\hfill\tt\color{nbframe-out-prompt}Out\hspace{4pt}{[}3{]}:\hspace{4pt}}\\*
                \vspace{-2.55\baselineskip}\begin{InvisibleVerbatim}
                \vspace{-0.5\baselineskip}
\begin{alltt}['\_\_add\_\_',
 '\_\_class\_\_',
 '\_\_contains\_\_',
 '\_\_delattr\_\_',
 '\_\_dir\_\_',
 '\_\_doc\_\_',
 '\_\_eq\_\_',
 '\_\_format\_\_',
 '\_\_ge\_\_',
 '\_\_getattribute\_\_',
 '\_\_getitem\_\_',
 '\_\_getnewargs\_\_',
 '\_\_gt\_\_',
 '\_\_hash\_\_',
 '\_\_init\_\_',
 '\_\_iter\_\_',
 '\_\_le\_\_',
 '\_\_len\_\_',
 '\_\_lt\_\_',
 '\_\_mod\_\_',
 '\_\_mul\_\_',
 '\_\_ne\_\_',
 '\_\_new\_\_',
 '\_\_reduce\_\_',
 '\_\_reduce\_ex\_\_',
 '\_\_repr\_\_',
 '\_\_rmod\_\_',
 '\_\_rmul\_\_',
 '\_\_setattr\_\_',
 '\_\_sizeof\_\_',
 '\_\_str\_\_',
 '\_\_subclasshook\_\_',
 'capitalize',
 'casefold',
 'center',
 'count',
 'encode',
 'endswith',
 'expandtabs',
 'find',
 'format',
 'format\_map',
 'index',
 'isalnum',
 'isalpha',
 'isdecimal',
 'isdigit',
 'isidentifier',
 'islower',
 'isnumeric',
 'isprintable',
 'isspace',
 'istitle',
 'isupper',
 'join',
 'ljust',
 'lower',
 'lstrip',
 'maketrans',
 'partition',
 'replace',
 'rfind',
 'rindex',
 'rjust',
 'rpartition',
 'rsplit',
 'rstrip',
 'split',
 'splitlines',
 'startswith',
 'strip',
 'swapcase',
 'title',
 'translate',
 'upper',
 'zfill']\end{alltt}

            \end{InvisibleVerbatim}
            
        
    
Algunos por ejemplo son:

    % Make sure that atleast 4 lines are below the HR
    \needspace{4\baselineskip}

    
        \vspace{6pt}
        \makebox[0.1\linewidth]{\smaller\hfill\tt\color{nbframe-in-prompt}In\hspace{4pt}{[}4{]}:\hspace{4pt}}\\*
        \vspace{-2.65\baselineskip}
        \begin{ColorVerbatim}
            \vspace{-0.7\baselineskip}
            \begin{Verbatim}[commandchars=\\\{\}]
\PY{n}{nombre}\PY{o}{.}\PY{n}{upper}\PY{p}{(}\PY{p}{)}
\end{Verbatim}

            
                \vspace{-0.2\baselineskip}
            
        \end{ColorVerbatim}
    

    

        % If the first block is an image, minipage the image.  Else
        % request a certain amount of space for the input text.
        \needspace{4\baselineskip}
        
        

            % Add document contents.
            
                \makebox[0.1\linewidth]{\smaller\hfill\tt\color{nbframe-out-prompt}Out\hspace{4pt}{[}4{]}:\hspace{4pt}}\\*
                \vspace{-2.55\baselineskip}\begin{InvisibleVerbatim}
                \vspace{-0.5\baselineskip}
\begin{alltt}'CATALINA'\end{alltt}

            \end{InvisibleVerbatim}
            
        
    


    % Make sure that atleast 4 lines are below the HR
    \needspace{4\baselineskip}

    
        \vspace{6pt}
        \makebox[0.1\linewidth]{\smaller\hfill\tt\color{nbframe-in-prompt}In\hspace{4pt}{[}5{]}:\hspace{4pt}}\\*
        \vspace{-2.65\baselineskip}
        \begin{ColorVerbatim}
            \vspace{-0.7\baselineskip}
            \begin{Verbatim}[commandchars=\\\{\}]
\PY{n}{nombre}\PY{o}{.}\PY{n}{lower}\PY{p}{(}\PY{p}{)}
\end{Verbatim}

            
                \vspace{-0.2\baselineskip}
            
        \end{ColorVerbatim}
    

    

        % If the first block is an image, minipage the image.  Else
        % request a certain amount of space for the input text.
        \needspace{4\baselineskip}
        
        

            % Add document contents.
            
                \makebox[0.1\linewidth]{\smaller\hfill\tt\color{nbframe-out-prompt}Out\hspace{4pt}{[}5{]}:\hspace{4pt}}\\*
                \vspace{-2.55\baselineskip}\begin{InvisibleVerbatim}
                \vspace{-0.5\baselineskip}
\begin{alltt}'catalina'\end{alltt}

            \end{InvisibleVerbatim}
            
        
    
Para saber que hace un determinado método podemos revisar la ayuda de
este.

    % Make sure that atleast 4 lines are below the HR
    \needspace{4\baselineskip}

    
        \vspace{6pt}
        \makebox[0.1\linewidth]{\smaller\hfill\tt\color{nbframe-in-prompt}In\hspace{4pt}{[}6{]}:\hspace{4pt}}\\*
        \vspace{-2.65\baselineskip}
        \begin{ColorVerbatim}
            \vspace{-0.7\baselineskip}
            \begin{Verbatim}[commandchars=\\\{\}]
\PY{n}{help}\PY{p}{(}\PY{n}{nombre}\PY{o}{.}\PY{n}{find}\PY{p}{)}
\end{Verbatim}

            
                \vspace{-0.2\baselineskip}
            
        \end{ColorVerbatim}
    

    

        % If the first block is an image, minipage the image.  Else
        % request a certain amount of space for the input text.
        \needspace{4\baselineskip}
        
        

            % Add document contents.
            
                \begin{InvisibleVerbatim}
                \vspace{-0.5\baselineskip}
\begin{alltt}Help on built-in function find:

find(\ldots)
    S.find(sub[, start[, end]]) -> int

    Return the lowest index in S where substring sub is found,
    such that sub is contained within S[start:end].  Optional
    arguments start and end are interpreted as in slice notation.

    Return -1 on failure.

\end{alltt}

            \end{InvisibleVerbatim}
            
        
    
Además una palabra es una cadena de caractéres (el tipo caracter no está
definido en Pyhton), es decir puedo tener nuevas subpalabras de la
palabra original

    % Make sure that atleast 4 lines are below the HR
    \needspace{4\baselineskip}

    
        \vspace{6pt}
        \makebox[0.1\linewidth]{\smaller\hfill\tt\color{nbframe-in-prompt}In\hspace{4pt}{[}7{]}:\hspace{4pt}}\\*
        \vspace{-2.65\baselineskip}
        \begin{ColorVerbatim}
            \vspace{-0.7\baselineskip}
            \begin{Verbatim}[commandchars=\\\{\}]
\PY{n}{nombre}\PY{p}{[}\PY{l+m+mi}{1}\PY{p}{]}
\end{Verbatim}

            
                \vspace{-0.2\baselineskip}
            
        \end{ColorVerbatim}
    

    

        % If the first block is an image, minipage the image.  Else
        % request a certain amount of space for the input text.
        \needspace{4\baselineskip}
        
        

            % Add document contents.
            
                \makebox[0.1\linewidth]{\smaller\hfill\tt\color{nbframe-out-prompt}Out\hspace{4pt}{[}7{]}:\hspace{4pt}}\\*
                \vspace{-2.55\baselineskip}\begin{InvisibleVerbatim}
                \vspace{-0.5\baselineskip}
\begin{alltt}'a'\end{alltt}

            \end{InvisibleVerbatim}
            
        
    


    % Make sure that atleast 4 lines are below the HR
    \needspace{4\baselineskip}

    
        \vspace{6pt}
        \makebox[0.1\linewidth]{\smaller\hfill\tt\color{nbframe-in-prompt}In\hspace{4pt}{[}8{]}:\hspace{4pt}}\\*
        \vspace{-2.65\baselineskip}
        \begin{ColorVerbatim}
            \vspace{-0.7\baselineskip}
            \begin{Verbatim}[commandchars=\\\{\}]
\PY{n}{nombre}\PY{p}{[}\PY{l+m+mi}{2}\PY{p}{:}\PY{l+m+mi}{5}\PY{p}{]}
\end{Verbatim}

            
                \vspace{-0.2\baselineskip}
            
        \end{ColorVerbatim}
    

    

        % If the first block is an image, minipage the image.  Else
        % request a certain amount of space for the input text.
        \needspace{4\baselineskip}
        
        

            % Add document contents.
            
                \makebox[0.1\linewidth]{\smaller\hfill\tt\color{nbframe-out-prompt}Out\hspace{4pt}{[}8{]}:\hspace{4pt}}\\*
                \vspace{-2.55\baselineskip}\begin{InvisibleVerbatim}
                \vspace{-0.5\baselineskip}
\begin{alltt}'tal'\end{alltt}

            \end{InvisibleVerbatim}
            
        
    


    % Make sure that atleast 4 lines are below the HR
    \needspace{4\baselineskip}

    
        \vspace{6pt}
        \makebox[0.1\linewidth]{\smaller\hfill\tt\color{nbframe-in-prompt}In\hspace{4pt}{[}9{]}:\hspace{4pt}}\\*
        \vspace{-2.65\baselineskip}
        \begin{ColorVerbatim}
            \vspace{-0.7\baselineskip}
            \begin{Verbatim}[commandchars=\\\{\}]
\PY{n}{nombre}\PY{p}{[}\PY{o}{\PYZhy{}}\PY{l+m+mi}{1}\PY{p}{]}
\end{Verbatim}

            
                \vspace{-0.2\baselineskip}
            
        \end{ColorVerbatim}
    

    

        % If the first block is an image, minipage the image.  Else
        % request a certain amount of space for the input text.
        \needspace{4\baselineskip}
        
        

            % Add document contents.
            
                \makebox[0.1\linewidth]{\smaller\hfill\tt\color{nbframe-out-prompt}Out\hspace{4pt}{[}9{]}:\hspace{4pt}}\\*
                \vspace{-2.55\baselineskip}\begin{InvisibleVerbatim}
                \vspace{-0.5\baselineskip}
\begin{alltt}'a'\end{alltt}

            \end{InvisibleVerbatim}
            
        
    


    % Make sure that atleast 4 lines are below the HR
    \needspace{4\baselineskip}

    
        \vspace{6pt}
        \makebox[0.1\linewidth]{\smaller\hfill\tt\color{nbframe-in-prompt}In\hspace{4pt}{[}10{]}:\hspace{4pt}}\\*
        \vspace{-2.65\baselineskip}
        \begin{ColorVerbatim}
            \vspace{-0.7\baselineskip}
            \begin{Verbatim}[commandchars=\\\{\}]
\PY{n}{nombre}\PY{p}{[}\PY{o}{\PYZhy{}}\PY{l+m+mi}{3}\PY{p}{:}\PY{p}{]}
\end{Verbatim}

            
                \vspace{-0.2\baselineskip}
            
        \end{ColorVerbatim}
    

    

        % If the first block is an image, minipage the image.  Else
        % request a certain amount of space for the input text.
        \needspace{4\baselineskip}
        
        

            % Add document contents.
            
                \makebox[0.1\linewidth]{\smaller\hfill\tt\color{nbframe-out-prompt}Out\hspace{4pt}{[}10{]}:\hspace{4pt}}\\*
                \vspace{-2.55\baselineskip}\begin{InvisibleVerbatim}
                \vspace{-0.5\baselineskip}
\begin{alltt}'ina'\end{alltt}

            \end{InvisibleVerbatim}
            
        
    


    % Make sure that atleast 4 lines are below the HR
    \needspace{4\baselineskip}

    
        \vspace{6pt}
        \makebox[0.1\linewidth]{\smaller\hfill\tt\color{nbframe-in-prompt}In\hspace{4pt}{[}11{]}:\hspace{4pt}}\\*
        \vspace{-2.65\baselineskip}
        \begin{ColorVerbatim}
            \vspace{-0.7\baselineskip}
            \begin{Verbatim}[commandchars=\\\{\}]
\PY{n}{nombre}\PY{p}{[}\PY{o}{\PYZhy{}}\PY{l+m+mi}{4}\PY{p}{:}\PY{o}{\PYZhy{}}\PY{l+m+mi}{1}\PY{p}{]}
\end{Verbatim}

            
                \vspace{-0.2\baselineskip}
            
        \end{ColorVerbatim}
    

    

        % If the first block is an image, minipage the image.  Else
        % request a certain amount of space for the input text.
        \needspace{4\baselineskip}
        
        

            % Add document contents.
            
                \makebox[0.1\linewidth]{\smaller\hfill\tt\color{nbframe-out-prompt}Out\hspace{4pt}{[}11{]}:\hspace{4pt}}\\*
                \vspace{-2.55\baselineskip}\begin{InvisibleVerbatim}
                \vspace{-0.5\baselineskip}
\begin{alltt}'lin'\end{alltt}

            \end{InvisibleVerbatim}
            
        
    
La notación \texttt{2:5} se lee de la siguiente manera: quiero el pedazo
(slide) del tercer caracter (ya que los índices comienzan en 0) hasta
antes del quinto espacio entre caractéres.

\emph{Ejercicio:} Investigue que hacen los demás métodos. Por ejemplo:

    % Make sure that atleast 4 lines are below the HR
    \needspace{4\baselineskip}

    
        \vspace{6pt}
        \makebox[0.1\linewidth]{\smaller\hfill\tt\color{nbframe-in-prompt}In\hspace{4pt}{[}12{]}:\hspace{4pt}}\\*
        \vspace{-2.65\baselineskip}
        \begin{ColorVerbatim}
            \vspace{-0.7\baselineskip}
            \begin{Verbatim}[commandchars=\\\{\}]
\PY{n}{help}\PY{p}{(}\PY{n}{nombre}\PY{o}{.}\PY{n}{join}\PY{p}{)}
\end{Verbatim}

            
                \vspace{-0.2\baselineskip}
            
        \end{ColorVerbatim}
    

    

        % If the first block is an image, minipage the image.  Else
        % request a certain amount of space for the input text.
        \needspace{4\baselineskip}
        
        

            % Add document contents.
            
                \begin{InvisibleVerbatim}
                \vspace{-0.5\baselineskip}
\begin{alltt}Help on built-in function join:

join(\ldots)
    S.join(iterable) -> str

    Return a string which is the concatenation of the strings in the
    iterable.  The separator between elements is S.

\end{alltt}

            \end{InvisibleVerbatim}
            
        
    


    % Make sure that atleast 4 lines are below the HR
    \needspace{4\baselineskip}

    
        \vspace{6pt}
        \makebox[0.1\linewidth]{\smaller\hfill\tt\color{nbframe-in-prompt}In\hspace{4pt}{[}13{]}:\hspace{4pt}}\\*
        \vspace{-2.65\baselineskip}
        \begin{ColorVerbatim}
            \vspace{-0.7\baselineskip}
            \begin{Verbatim}[commandchars=\\\{\}]
\PY{n}{nombre\PYZus{}completo} \PY{o}{=} \PY{l+s}{\PYZsq{}}\PY{l+s}{ }\PY{l+s}{\PYZsq{}}\PY{o}{.}\PY{n}{join}\PY{p}{(}\PY{p}{(}\PY{n}{nombre}\PY{p}{,} \PY{n}{apellido}\PY{p}{)}\PY{p}{)}
\end{Verbatim}

            
                \vspace{-0.2\baselineskip}
            
        \end{ColorVerbatim}
    


    % Make sure that atleast 4 lines are below the HR
    \needspace{4\baselineskip}

    
        \vspace{6pt}
        \makebox[0.1\linewidth]{\smaller\hfill\tt\color{nbframe-in-prompt}In\hspace{4pt}{[}14{]}:\hspace{4pt}}\\*
        \vspace{-2.65\baselineskip}
        \begin{ColorVerbatim}
            \vspace{-0.7\baselineskip}
            \begin{Verbatim}[commandchars=\\\{\}]
\PY{k}{print}\PY{p}{(}\PY{n}{nombre\PYZus{}completo}\PY{p}{)}
\end{Verbatim}

            
                \vspace{-0.2\baselineskip}
            
        \end{ColorVerbatim}
    

    

        % If the first block is an image, minipage the image.  Else
        % request a certain amount of space for the input text.
        \needspace{4\baselineskip}
        
        

            % Add document contents.
            
                \begin{InvisibleVerbatim}
                \vspace{-0.5\baselineskip}
\begin{alltt}Catalina Valderrama
\end{alltt}

            \end{InvisibleVerbatim}
            
        
    
\emph{Observación:} Muchas veces nos encontraremos con instrucciones
como esta

    % Make sure that atleast 4 lines are below the HR
    \needspace{4\baselineskip}

    
        \vspace{6pt}
        \makebox[0.1\linewidth]{\smaller\hfill\tt\color{nbframe-in-prompt}In\hspace{4pt}{[}15{]}:\hspace{4pt}}\\*
        \vspace{-2.65\baselineskip}
        \begin{ColorVerbatim}
            \vspace{-0.7\baselineskip}
            \begin{Verbatim}[commandchars=\\\{\}]
\PY{k}{print}\PY{p}{(}\PY{l+s}{\PYZsq{}}\PY{l+s}{Los hijos de }\PY{l+s+si}{\PYZpc{}s}\PY{l+s}{ tendrán como apellido }\PY{l+s+si}{\PYZpc{}s}\PY{l+s}{\PYZsq{}}\PY{o}{\PYZpc{}} \PY{p}{(}\PY{n}{nombre}\PY{p}{,} \PY{n}{apellido}\PY{p}{)}\PY{p}{)}
\end{Verbatim}

            
                \vspace{-0.2\baselineskip}
            
        \end{ColorVerbatim}
    

    

        % If the first block is an image, minipage the image.  Else
        % request a certain amount of space for the input text.
        \needspace{4\baselineskip}
        
        

            % Add document contents.
            
                \begin{InvisibleVerbatim}
                \vspace{-0.5\baselineskip}
\begin{alltt}Los hijos de Catalina tendrán como apellido Valderrama
\end{alltt}

            \end{InvisibleVerbatim}
            
        
    
Aún cuando se produce un resultado deseado su uso no es muy
\emph{Pythonista}. En su lugar debemos de usar el método \texttt{format}
para palabras.

    % Make sure that atleast 4 lines are below the HR
    \needspace{4\baselineskip}

    
        \vspace{6pt}
        \makebox[0.1\linewidth]{\smaller\hfill\tt\color{nbframe-in-prompt}In\hspace{4pt}{[}16{]}:\hspace{4pt}}\\*
        \vspace{-2.65\baselineskip}
        \begin{ColorVerbatim}
            \vspace{-0.7\baselineskip}
            \begin{Verbatim}[commandchars=\\\{\}]
\PY{k}{print}\PY{p}{(}\PY{l+s}{\PYZsq{}}\PY{l+s}{Los hijos de \PYZob{}nombre\PYZcb{} tendrán como apellido \PYZob{}apellido\PYZcb{}}\PY{l+s}{\PYZsq{}}\PY{o}{.}\PY{n}{format}\PY{p}{(}\PY{n}{nombre}\PY{o}{=}\PY{n}{nombre}\PY{p}{,} \PY{n}{apellido}\PY{o}{=}\PY{n}{apellido}\PY{p}{)}\PY{p}{)}
\end{Verbatim}

            
                \vspace{-0.2\baselineskip}
            
        \end{ColorVerbatim}
    

    

        % If the first block is an image, minipage the image.  Else
        % request a certain amount of space for the input text.
        \needspace{4\baselineskip}
        
        

            % Add document contents.
            
                \begin{InvisibleVerbatim}
                \vspace{-0.5\baselineskip}
\begin{alltt}Los hijos de Catalina tendrán como apellido Valderrama
\end{alltt}

            \end{InvisibleVerbatim}
            
        
    
\subsection{Listas}\label{listas}

A diferencia de los tipos de datos como enteros, palabras, flotantes o
booleanos, las listas son una estructura de datos, es decir es una forma
de organizar los datos anteriores de tal manera que sea más fácil
manipularlas.

Es importante decir que los elementos de una lista pueden ser de
cualquier tipo, inclusive listas.

    % Make sure that atleast 4 lines are below the HR
    \needspace{4\baselineskip}

    
        \vspace{6pt}
        \makebox[0.1\linewidth]{\smaller\hfill\tt\color{nbframe-in-prompt}In\hspace{4pt}{[}17{]}:\hspace{4pt}}\\*
        \vspace{-2.65\baselineskip}
        \begin{ColorVerbatim}
            \vspace{-0.7\baselineskip}
            \begin{Verbatim}[commandchars=\\\{\}]
\PY{n}{mi\PYZus{}lista} \PY{o}{=} \PY{p}{[}\PY{l+m+mi}{1}\PY{p}{,} \PY{l+m+mf}{3.5}\PY{p}{,} \PY{n+nb+bp}{True}\PY{p}{,} \PY{p}{[}\PY{l+s}{\PYZsq{}}\PY{l+s}{a}\PY{l+s}{\PYZsq{}}\PY{p}{,} \PY{l+s}{\PYZsq{}}\PY{l+s}{b}\PY{l+s}{\PYZsq{}}\PY{p}{,} \PY{l+s}{\PYZsq{}}\PY{l+s}{c}\PY{l+s}{\PYZsq{}}\PY{p}{]}\PY{p}{]}
\end{Verbatim}

            
                \vspace{-0.2\baselineskip}
            
        \end{ColorVerbatim}
    


    % Make sure that atleast 4 lines are below the HR
    \needspace{4\baselineskip}

    
        \vspace{6pt}
        \makebox[0.1\linewidth]{\smaller\hfill\tt\color{nbframe-in-prompt}In\hspace{4pt}{[}18{]}:\hspace{4pt}}\\*
        \vspace{-2.65\baselineskip}
        \begin{ColorVerbatim}
            \vspace{-0.7\baselineskip}
            \begin{Verbatim}[commandchars=\\\{\}]
\PY{n+nb}{type}\PY{p}{(}\PY{n}{mi\PYZus{}lista}\PY{p}{)}
\end{Verbatim}

            
                \vspace{-0.2\baselineskip}
            
        \end{ColorVerbatim}
    

    

        % If the first block is an image, minipage the image.  Else
        % request a certain amount of space for the input text.
        \needspace{4\baselineskip}
        
        

            % Add document contents.
            
                \makebox[0.1\linewidth]{\smaller\hfill\tt\color{nbframe-out-prompt}Out\hspace{4pt}{[}18{]}:\hspace{4pt}}\\*
                \vspace{-2.55\baselineskip}\begin{InvisibleVerbatim}
                \vspace{-0.5\baselineskip}
\begin{alltt}builtins.list\end{alltt}

            \end{InvisibleVerbatim}
            
        
    


    % Make sure that atleast 4 lines are below the HR
    \needspace{4\baselineskip}

    
        \vspace{6pt}
        \makebox[0.1\linewidth]{\smaller\hfill\tt\color{nbframe-in-prompt}In\hspace{4pt}{[}19{]}:\hspace{4pt}}\\*
        \vspace{-2.65\baselineskip}
        \begin{ColorVerbatim}
            \vspace{-0.7\baselineskip}
            \begin{Verbatim}[commandchars=\\\{\}]
\PY{n+nb}{dir}\PY{p}{(}\PY{n}{mi\PYZus{}lista}\PY{p}{)}
\end{Verbatim}

            
                \vspace{-0.2\baselineskip}
            
        \end{ColorVerbatim}
    

    

        % If the first block is an image, minipage the image.  Else
        % request a certain amount of space for the input text.
        \needspace{4\baselineskip}
        
        

            % Add document contents.
            
                \makebox[0.1\linewidth]{\smaller\hfill\tt\color{nbframe-out-prompt}Out\hspace{4pt}{[}19{]}:\hspace{4pt}}\\*
                \vspace{-2.55\baselineskip}\begin{InvisibleVerbatim}
                \vspace{-0.5\baselineskip}
\begin{alltt}['\_\_add\_\_',
 '\_\_class\_\_',
 '\_\_contains\_\_',
 '\_\_delattr\_\_',
 '\_\_delitem\_\_',
 '\_\_dir\_\_',
 '\_\_doc\_\_',
 '\_\_eq\_\_',
 '\_\_format\_\_',
 '\_\_ge\_\_',
 '\_\_getattribute\_\_',
 '\_\_getitem\_\_',
 '\_\_gt\_\_',
 '\_\_hash\_\_',
 '\_\_iadd\_\_',
 '\_\_imul\_\_',
 '\_\_init\_\_',
 '\_\_iter\_\_',
 '\_\_le\_\_',
 '\_\_len\_\_',
 '\_\_lt\_\_',
 '\_\_mul\_\_',
 '\_\_ne\_\_',
 '\_\_new\_\_',
 '\_\_reduce\_\_',
 '\_\_reduce\_ex\_\_',
 '\_\_repr\_\_',
 '\_\_reversed\_\_',
 '\_\_rmul\_\_',
 '\_\_setattr\_\_',
 '\_\_setitem\_\_',
 '\_\_sizeof\_\_',
 '\_\_str\_\_',
 '\_\_subclasshook\_\_',
 'append',
 'clear',
 'copy',
 'count',
 'extend',
 'index',
 'insert',
 'pop',
 'remove',
 'reverse',
 'sort']\end{alltt}

            \end{InvisibleVerbatim}
            
        
    
Es importante hacer enfasis en que para una lista el orden de los
elementos importa. Además una manera de ver lo que es una lista es como
imaginarse que es una pila de papeles, inclusive muchos de los métodos
para listas son tarea que normalmente se hace con las pilas de papeles

    % Make sure that atleast 4 lines are below the HR
    \needspace{4\baselineskip}

    
        \vspace{6pt}
        \makebox[0.1\linewidth]{\smaller\hfill\tt\color{nbframe-in-prompt}In\hspace{4pt}{[}20{]}:\hspace{4pt}}\\*
        \vspace{-2.65\baselineskip}
        \begin{ColorVerbatim}
            \vspace{-0.7\baselineskip}
            \begin{Verbatim}[commandchars=\\\{\}]
\PY{p}{[}\PY{l+m+mi}{1}\PY{p}{,} \PY{l+m+mi}{2}\PY{p}{,} \PY{l+m+mi}{3}\PY{p}{,} \PY{l+m+mi}{4}\PY{p}{]} \PY{o}{==} \PY{p}{[}\PY{l+m+mi}{1}\PY{p}{,} \PY{l+m+mi}{2}\PY{p}{,} \PY{l+m+mi}{4}\PY{p}{,} \PY{l+m+mi}{3}\PY{p}{]}
\end{Verbatim}

            
                \vspace{-0.2\baselineskip}
            
        \end{ColorVerbatim}
    

    

        % If the first block is an image, minipage the image.  Else
        % request a certain amount of space for the input text.
        \needspace{4\baselineskip}
        
        

            % Add document contents.
            
                \makebox[0.1\linewidth]{\smaller\hfill\tt\color{nbframe-out-prompt}Out\hspace{4pt}{[}20{]}:\hspace{4pt}}\\*
                \vspace{-2.55\baselineskip}\begin{InvisibleVerbatim}
                \vspace{-0.5\baselineskip}
\begin{alltt}False\end{alltt}

            \end{InvisibleVerbatim}
            
        
    


    % Make sure that atleast 4 lines are below the HR
    \needspace{4\baselineskip}

    
        \vspace{6pt}
        \makebox[0.1\linewidth]{\smaller\hfill\tt\color{nbframe-in-prompt}In\hspace{4pt}{[}21{]}:\hspace{4pt}}\\*
        \vspace{-2.65\baselineskip}
        \begin{ColorVerbatim}
            \vspace{-0.7\baselineskip}
            \begin{Verbatim}[commandchars=\\\{\}]
\PY{n}{lista\PYZus{}numeros} \PY{o}{=} \PY{p}{[}\PY{l+m+mi}{1}\PY{p}{,} \PY{l+m+mi}{2}\PY{p}{,} \PY{l+m+mi}{3}\PY{p}{,} \PY{l+m+mi}{4}\PY{p}{,} \PY{l+m+mi}{5}\PY{p}{,} \PY{l+m+mi}{6}\PY{p}{,} \PY{l+m+mi}{7}\PY{p}{]}
\end{Verbatim}

            
                \vspace{-0.2\baselineskip}
            
        \end{ColorVerbatim}
    


    % Make sure that atleast 4 lines are below the HR
    \needspace{4\baselineskip}

    
        \vspace{6pt}
        \makebox[0.1\linewidth]{\smaller\hfill\tt\color{nbframe-in-prompt}In\hspace{4pt}{[}22{]}:\hspace{4pt}}\\*
        \vspace{-2.65\baselineskip}
        \begin{ColorVerbatim}
            \vspace{-0.7\baselineskip}
            \begin{Verbatim}[commandchars=\\\{\}]
\PY{k}{print}\PY{p}{(}\PY{n}{lista\PYZus{}numeros}\PY{p}{)}
\end{Verbatim}

            
                \vspace{-0.2\baselineskip}
            
        \end{ColorVerbatim}
    

    

        % If the first block is an image, minipage the image.  Else
        % request a certain amount of space for the input text.
        \needspace{4\baselineskip}
        
        

            % Add document contents.
            
                \begin{InvisibleVerbatim}
                \vspace{-0.5\baselineskip}
\begin{alltt}[1, 2, 3, 4, 5, 6, 7]
\end{alltt}

            \end{InvisibleVerbatim}
            
        
    


    % Make sure that atleast 4 lines are below the HR
    \needspace{4\baselineskip}

    
        \vspace{6pt}
        \makebox[0.1\linewidth]{\smaller\hfill\tt\color{nbframe-in-prompt}In\hspace{4pt}{[}23{]}:\hspace{4pt}}\\*
        \vspace{-2.65\baselineskip}
        \begin{ColorVerbatim}
            \vspace{-0.7\baselineskip}
            \begin{Verbatim}[commandchars=\\\{\}]
\PY{n}{lista\PYZus{}numeros}\PY{o}{.}\PY{n}{pop}\PY{p}{(}\PY{l+m+mi}{0}\PY{p}{)}
\end{Verbatim}

            
                \vspace{-0.2\baselineskip}
            
        \end{ColorVerbatim}
    

    

        % If the first block is an image, minipage the image.  Else
        % request a certain amount of space for the input text.
        \needspace{4\baselineskip}
        
        

            % Add document contents.
            
                \makebox[0.1\linewidth]{\smaller\hfill\tt\color{nbframe-out-prompt}Out\hspace{4pt}{[}23{]}:\hspace{4pt}}\\*
                \vspace{-2.55\baselineskip}\begin{InvisibleVerbatim}
                \vspace{-0.5\baselineskip}
\begin{alltt}1\end{alltt}

            \end{InvisibleVerbatim}
            
        
    


    % Make sure that atleast 4 lines are below the HR
    \needspace{4\baselineskip}

    
        \vspace{6pt}
        \makebox[0.1\linewidth]{\smaller\hfill\tt\color{nbframe-in-prompt}In\hspace{4pt}{[}24{]}:\hspace{4pt}}\\*
        \vspace{-2.65\baselineskip}
        \begin{ColorVerbatim}
            \vspace{-0.7\baselineskip}
            \begin{Verbatim}[commandchars=\\\{\}]
\PY{k}{print}\PY{p}{(}\PY{n}{lista\PYZus{}numeros}\PY{p}{)}
\end{Verbatim}

            
                \vspace{-0.2\baselineskip}
            
        \end{ColorVerbatim}
    

    

        % If the first block is an image, minipage the image.  Else
        % request a certain amount of space for the input text.
        \needspace{4\baselineskip}
        
        

            % Add document contents.
            
                \begin{InvisibleVerbatim}
                \vspace{-0.5\baselineskip}
\begin{alltt}[2, 3, 4, 5, 6, 7]
\end{alltt}

            \end{InvisibleVerbatim}
            
        
    


    % Make sure that atleast 4 lines are below the HR
    \needspace{4\baselineskip}

    
        \vspace{6pt}
        \makebox[0.1\linewidth]{\smaller\hfill\tt\color{nbframe-in-prompt}In\hspace{4pt}{[}25{]}:\hspace{4pt}}\\*
        \vspace{-2.65\baselineskip}
        \begin{ColorVerbatim}
            \vspace{-0.7\baselineskip}
            \begin{Verbatim}[commandchars=\\\{\}]
\PY{n}{lista\PYZus{}numeros}\PY{o}{.}\PY{n}{append}\PY{p}{(}\PY{l+m+mi}{1}\PY{p}{)}
\end{Verbatim}

            
                \vspace{-0.2\baselineskip}
            
        \end{ColorVerbatim}
    


    % Make sure that atleast 4 lines are below the HR
    \needspace{4\baselineskip}

    
        \vspace{6pt}
        \makebox[0.1\linewidth]{\smaller\hfill\tt\color{nbframe-in-prompt}In\hspace{4pt}{[}26{]}:\hspace{4pt}}\\*
        \vspace{-2.65\baselineskip}
        \begin{ColorVerbatim}
            \vspace{-0.7\baselineskip}
            \begin{Verbatim}[commandchars=\\\{\}]
\PY{k}{print}\PY{p}{(}\PY{n}{lista\PYZus{}numeros}\PY{p}{)}
\end{Verbatim}

            
                \vspace{-0.2\baselineskip}
            
        \end{ColorVerbatim}
    

    

        % If the first block is an image, minipage the image.  Else
        % request a certain amount of space for the input text.
        \needspace{4\baselineskip}
        
        

            % Add document contents.
            
                \begin{InvisibleVerbatim}
                \vspace{-0.5\baselineskip}
\begin{alltt}[2, 3, 4, 5, 6, 7, 1]
\end{alltt}

            \end{InvisibleVerbatim}
            
        
    


    % Make sure that atleast 4 lines are below the HR
    \needspace{4\baselineskip}

    
        \vspace{6pt}
        \makebox[0.1\linewidth]{\smaller\hfill\tt\color{nbframe-in-prompt}In\hspace{4pt}{[}27{]}:\hspace{4pt}}\\*
        \vspace{-2.65\baselineskip}
        \begin{ColorVerbatim}
            \vspace{-0.7\baselineskip}
            \begin{Verbatim}[commandchars=\\\{\}]
\PY{n}{lista\PYZus{}numeros}\PY{o}{.}\PY{n}{sort}\PY{p}{(}\PY{p}{)}
\end{Verbatim}

            
                \vspace{-0.2\baselineskip}
            
        \end{ColorVerbatim}
    


    % Make sure that atleast 4 lines are below the HR
    \needspace{4\baselineskip}

    
        \vspace{6pt}
        \makebox[0.1\linewidth]{\smaller\hfill\tt\color{nbframe-in-prompt}In\hspace{4pt}{[}28{]}:\hspace{4pt}}\\*
        \vspace{-2.65\baselineskip}
        \begin{ColorVerbatim}
            \vspace{-0.7\baselineskip}
            \begin{Verbatim}[commandchars=\\\{\}]
\PY{k}{print}\PY{p}{(}\PY{n}{lista\PYZus{}numeros}\PY{p}{)}
\end{Verbatim}

            
                \vspace{-0.2\baselineskip}
            
        \end{ColorVerbatim}
    

    

        % If the first block is an image, minipage the image.  Else
        % request a certain amount of space for the input text.
        \needspace{4\baselineskip}
        
        

            % Add document contents.
            
                \begin{InvisibleVerbatim}
                \vspace{-0.5\baselineskip}
\begin{alltt}[1, 2, 3, 4, 5, 6, 7]
\end{alltt}

            \end{InvisibleVerbatim}
            
        
    


    % Make sure that atleast 4 lines are below the HR
    \needspace{4\baselineskip}

    
        \vspace{6pt}
        \makebox[0.1\linewidth]{\smaller\hfill\tt\color{nbframe-in-prompt}In\hspace{4pt}{[}29{]}:\hspace{4pt}}\\*
        \vspace{-2.65\baselineskip}
        \begin{ColorVerbatim}
            \vspace{-0.7\baselineskip}
            \begin{Verbatim}[commandchars=\\\{\}]
\PY{n}{lista\PYZus{}numeros}\PY{o}{.}\PY{n}{extend}\PY{p}{(}\PY{p}{[}\PY{l+m+mi}{3}\PY{p}{,} \PY{l+m+mi}{3}\PY{p}{,} \PY{l+m+mi}{4}\PY{p}{,} \PY{l+m+mi}{6}\PY{p}{,} \PY{l+m+mi}{7}\PY{p}{,} \PY{l+m+mi}{9}\PY{p}{]}\PY{p}{)}
\end{Verbatim}

            
                \vspace{-0.2\baselineskip}
            
        \end{ColorVerbatim}
    


    % Make sure that atleast 4 lines are below the HR
    \needspace{4\baselineskip}

    
        \vspace{6pt}
        \makebox[0.1\linewidth]{\smaller\hfill\tt\color{nbframe-in-prompt}In\hspace{4pt}{[}30{]}:\hspace{4pt}}\\*
        \vspace{-2.65\baselineskip}
        \begin{ColorVerbatim}
            \vspace{-0.7\baselineskip}
            \begin{Verbatim}[commandchars=\\\{\}]
\PY{n}{lista\PYZus{}numeros}\PY{o}{.}\PY{n}{count}\PY{p}{(}\PY{l+m+mi}{3}\PY{p}{)}
\end{Verbatim}

            
                \vspace{-0.2\baselineskip}
            
        \end{ColorVerbatim}
    

    

        % If the first block is an image, minipage the image.  Else
        % request a certain amount of space for the input text.
        \needspace{4\baselineskip}
        
        

            % Add document contents.
            
                \makebox[0.1\linewidth]{\smaller\hfill\tt\color{nbframe-out-prompt}Out\hspace{4pt}{[}30{]}:\hspace{4pt}}\\*
                \vspace{-2.55\baselineskip}\begin{InvisibleVerbatim}
                \vspace{-0.5\baselineskip}
\begin{alltt}3\end{alltt}

            \end{InvisibleVerbatim}
            
        
    


    % Make sure that atleast 4 lines are below the HR
    \needspace{4\baselineskip}

    
        \vspace{6pt}
        \makebox[0.1\linewidth]{\smaller\hfill\tt\color{nbframe-in-prompt}In\hspace{4pt}{[}31{]}:\hspace{4pt}}\\*
        \vspace{-2.65\baselineskip}
        \begin{ColorVerbatim}
            \vspace{-0.7\baselineskip}
            \begin{Verbatim}[commandchars=\\\{\}]
\PY{k}{print}\PY{p}{(}\PY{n}{lista\PYZus{}numeros}\PY{p}{)}
\end{Verbatim}

            
                \vspace{-0.2\baselineskip}
            
        \end{ColorVerbatim}
    

    

        % If the first block is an image, minipage the image.  Else
        % request a certain amount of space for the input text.
        \needspace{4\baselineskip}
        
        

            % Add document contents.
            
                \begin{InvisibleVerbatim}
                \vspace{-0.5\baselineskip}
\begin{alltt}[1, 2, 3, 4, 5, 6, 7, 3, 3, 4, 6, 7, 9]
\end{alltt}

            \end{InvisibleVerbatim}
            
        
    
\emph{Ejercicio:} ¿Cómo funciona el método \texttt{sort} si la lista
tiene números y letras?Otra propiedad importante que tienen las listas es que son iterables.

    % Make sure that atleast 4 lines are below the HR
    \needspace{4\baselineskip}

    
        \vspace{6pt}
        \makebox[0.1\linewidth]{\smaller\hfill\tt\color{nbframe-in-prompt}In\hspace{4pt}{[}32{]}:\hspace{4pt}}\\*
        \vspace{-2.65\baselineskip}
        \begin{ColorVerbatim}
            \vspace{-0.7\baselineskip}
            \begin{Verbatim}[commandchars=\\\{\}]
\PY{n}{lista\PYZus{}de\PYZus{}precios\PYZus{}siniva} \PY{o}{=} \PY{p}{[}\PY{l+m+mi}{12}\PY{p}{,} \PY{l+m+mi}{34}\PY{p}{,} \PY{l+m+mi}{56}\PY{p}{,} \PY{l+m+mi}{23}\PY{p}{,} \PY{l+m+mi}{75}\PY{p}{,} \PY{l+m+mi}{92}\PY{p}{,} \PY{l+m+mi}{45}\PY{p}{,} \PY{l+m+mi}{97}\PY{p}{]}
\end{Verbatim}

            
                \vspace{-0.2\baselineskip}
            
        \end{ColorVerbatim}
    


    % Make sure that atleast 4 lines are below the HR
    \needspace{4\baselineskip}

    
        \vspace{6pt}
        \makebox[0.1\linewidth]{\smaller\hfill\tt\color{nbframe-in-prompt}In\hspace{4pt}{[}33{]}:\hspace{4pt}}\\*
        \vspace{-2.65\baselineskip}
        \begin{ColorVerbatim}
            \vspace{-0.7\baselineskip}
            \begin{Verbatim}[commandchars=\\\{\}]
\PY{k}{for} \PY{n}{precio} \PY{o+ow}{in} \PY{n}{lista\PYZus{}de\PYZus{}precios\PYZus{}siniva}\PY{p}{:}
    \PY{k}{print}\PY{p}{(}\PY{l+s}{\PYZsq{}}\PY{l+s}{El precio del producto es \PYZdl{}\PYZob{}precio:.2f\PYZcb{}}\PY{l+s}{\PYZsq{}}\PY{o}{.}\PY{n}{format}\PY{p}{(}\PY{n}{precio}\PY{o}{=}\PY{n}{precio}\PY{o}{*}\PY{l+m+mf}{1.16}\PY{p}{)}\PY{p}{)}
\end{Verbatim}

            
                \vspace{-0.2\baselineskip}
            
        \end{ColorVerbatim}
    

    

        % If the first block is an image, minipage the image.  Else
        % request a certain amount of space for the input text.
        \needspace{4\baselineskip}
        
        

            % Add document contents.
            
                \begin{InvisibleVerbatim}
                \vspace{-0.5\baselineskip}
\begin{alltt}El precio del producto es \$13.92
El precio del producto es \$39.44
El precio del producto es \$64.96
El precio del producto es \$26.68
El precio del producto es \$87.00
El precio del producto es \$106.72
El precio del producto es \$52.20
El precio del producto es \$112.52
\end{alltt}

            \end{InvisibleVerbatim}
            
        
    
\emph{Ejercicio:} ¿Qué hace el siguiente código

\begin{verbatim}
for indice, precio in enumerate(lista_de_precios_siniva):
    print('El precio del producto (numero) es ${precio:.2f}'.format(numero=indice, precio=precio*1.16))
\end{verbatim}

?

¿Estás seguro?. Puedes corregir el código para que haga lo correcto.

Una de las funcionalidades más usadas en las lista es la compresión de
las mismas (\emph{list comprehension}). Esto es la capacidad de generar
nuevas listas de una lista original.

    % Make sure that atleast 4 lines are below the HR
    \needspace{4\baselineskip}

    
        \vspace{6pt}
        \makebox[0.1\linewidth]{\smaller\hfill\tt\color{nbframe-in-prompt}In\hspace{4pt}{[}34{]}:\hspace{4pt}}\\*
        \vspace{-2.65\baselineskip}
        \begin{ColorVerbatim}
            \vspace{-0.7\baselineskip}
            \begin{Verbatim}[commandchars=\\\{\}]
\PY{n}{numeros} \PY{o}{=} \PY{p}{[}\PY{l+m+mi}{1}\PY{p}{,} \PY{l+m+mi}{2}\PY{p}{,} \PY{l+m+mi}{3}\PY{p}{,} \PY{l+m+mi}{4}\PY{p}{,} \PY{l+m+mi}{5}\PY{p}{,} \PY{l+m+mi}{6}\PY{p}{,} \PY{l+m+mi}{7}\PY{p}{,} \PY{l+m+mi}{8}\PY{p}{,} \PY{l+m+mi}{9}\PY{p}{,} \PY{l+m+mi}{10}\PY{p}{]}
\end{Verbatim}

            
                \vspace{-0.2\baselineskip}
            
        \end{ColorVerbatim}
    


    % Make sure that atleast 4 lines are below the HR
    \needspace{4\baselineskip}

    
        \vspace{6pt}
        \makebox[0.1\linewidth]{\smaller\hfill\tt\color{nbframe-in-prompt}In\hspace{4pt}{[}35{]}:\hspace{4pt}}\\*
        \vspace{-2.65\baselineskip}
        \begin{ColorVerbatim}
            \vspace{-0.7\baselineskip}
            \begin{Verbatim}[commandchars=\\\{\}]
\PY{n}{cuadrado\PYZus{}primeros} \PY{o}{=} \PY{p}{[}\PY{n}{numero}\PY{o}{*}\PY{o}{*}\PY{l+m+mi}{2} \PY{k}{for} \PY{n}{numero} \PY{o+ow}{in} \PY{n}{numeros}\PY{p}{]}
\end{Verbatim}

            
                \vspace{-0.2\baselineskip}
            
        \end{ColorVerbatim}
    


    % Make sure that atleast 4 lines are below the HR
    \needspace{4\baselineskip}

    
        \vspace{6pt}
        \makebox[0.1\linewidth]{\smaller\hfill\tt\color{nbframe-in-prompt}In\hspace{4pt}{[}36{]}:\hspace{4pt}}\\*
        \vspace{-2.65\baselineskip}
        \begin{ColorVerbatim}
            \vspace{-0.7\baselineskip}
            \begin{Verbatim}[commandchars=\\\{\}]
\PY{n}{cuadrado\PYZus{}primeros}
\end{Verbatim}

            
                \vspace{-0.2\baselineskip}
            
        \end{ColorVerbatim}
    

    

        % If the first block is an image, minipage the image.  Else
        % request a certain amount of space for the input text.
        \needspace{4\baselineskip}
        
        

            % Add document contents.
            
                \makebox[0.1\linewidth]{\smaller\hfill\tt\color{nbframe-out-prompt}Out\hspace{4pt}{[}36{]}:\hspace{4pt}}\\*
                \vspace{-2.55\baselineskip}\begin{InvisibleVerbatim}
                \vspace{-0.5\baselineskip}
\begin{alltt}[1, 4, 9, 16, 25, 36, 49, 64, 81, 100]\end{alltt}

            \end{InvisibleVerbatim}
            
        
    


    % Make sure that atleast 4 lines are below the HR
    \needspace{4\baselineskip}

    
        \vspace{6pt}
        \makebox[0.1\linewidth]{\smaller\hfill\tt\color{nbframe-in-prompt}In\hspace{4pt}{[}37{]}:\hspace{4pt}}\\*
        \vspace{-2.65\baselineskip}
        \begin{ColorVerbatim}
            \vspace{-0.7\baselineskip}
            \begin{Verbatim}[commandchars=\\\{\}]
\PY{n}{impares} \PY{o}{=} \PY{p}{[}\PY{n}{numero} \PY{k}{for} \PY{n}{numero} \PY{o+ow}{in} \PY{n}{numeros} \PY{k}{if} \PY{n}{numero}\PY{o}{\PYZpc{}}\PY{k}{2}\PY{o}{==}\PY{l+m+mi}{1}\PY{p}{]}
\end{Verbatim}

            
                \vspace{-0.2\baselineskip}
            
        \end{ColorVerbatim}
    


    % Make sure that atleast 4 lines are below the HR
    \needspace{4\baselineskip}

    
        \vspace{6pt}
        \makebox[0.1\linewidth]{\smaller\hfill\tt\color{nbframe-in-prompt}In\hspace{4pt}{[}38{]}:\hspace{4pt}}\\*
        \vspace{-2.65\baselineskip}
        \begin{ColorVerbatim}
            \vspace{-0.7\baselineskip}
            \begin{Verbatim}[commandchars=\\\{\}]
\PY{n}{impares}
\end{Verbatim}

            
                \vspace{-0.2\baselineskip}
            
        \end{ColorVerbatim}
    

    

        % If the first block is an image, minipage the image.  Else
        % request a certain amount of space for the input text.
        \needspace{4\baselineskip}
        
        

            % Add document contents.
            
                \makebox[0.1\linewidth]{\smaller\hfill\tt\color{nbframe-out-prompt}Out\hspace{4pt}{[}38{]}:\hspace{4pt}}\\*
                \vspace{-2.55\baselineskip}\begin{InvisibleVerbatim}
                \vspace{-0.5\baselineskip}
\begin{alltt}[1, 3, 5, 7, 9]\end{alltt}

            \end{InvisibleVerbatim}
            
        
    


    % Make sure that atleast 4 lines are below the HR
    \needspace{4\baselineskip}

    
        \vspace{6pt}
        \makebox[0.1\linewidth]{\smaller\hfill\tt\color{nbframe-in-prompt}In\hspace{4pt}{[}39{]}:\hspace{4pt}}\\*
        \vspace{-2.65\baselineskip}
        \begin{ColorVerbatim}
            \vspace{-0.7\baselineskip}
            \begin{Verbatim}[commandchars=\\\{\}]
\PY{n}{nombres} \PY{o}{=} \PY{p}{[}\PY{l+s}{\PYZsq{}}\PY{l+s}{Juan}\PY{l+s}{\PYZsq{}}\PY{p}{,} \PY{l+s}{\PYZsq{}}\PY{l+s}{Jaime}\PY{l+s}{\PYZsq{}}\PY{p}{,} \PY{l+s}{\PYZsq{}}\PY{l+s}{Pablo}\PY{l+s}{\PYZsq{}}\PY{p}{,} \PY{l+s}{\PYZsq{}}\PY{l+s}{Lucero}\PY{l+s}{\PYZsq{}}\PY{p}{,} \PY{l+s}{\PYZsq{}}\PY{l+s}{Jimena}\PY{l+s}{\PYZsq{}}\PY{p}{,} \PY{l+s}{\PYZsq{}}\PY{l+s}{Marco}\PY{l+s}{\PYZsq{}}\PY{p}{]}
\end{Verbatim}

            
                \vspace{-0.2\baselineskip}
            
        \end{ColorVerbatim}
    


    % Make sure that atleast 4 lines are below the HR
    \needspace{4\baselineskip}

    
        \vspace{6pt}
        \makebox[0.1\linewidth]{\smaller\hfill\tt\color{nbframe-in-prompt}In\hspace{4pt}{[}40{]}:\hspace{4pt}}\\*
        \vspace{-2.65\baselineskip}
        \begin{ColorVerbatim}
            \vspace{-0.7\baselineskip}
            \begin{Verbatim}[commandchars=\\\{\}]
\PY{n}{nombres\PYZus{}conj} \PY{o}{=} \PY{p}{[}\PY{n}{nombre} \PY{k}{for} \PY{n}{nombre} \PY{o+ow}{in} \PY{n}{nombres} \PY{k}{if} \PY{n}{nombre}\PY{o}{.}\PY{n}{startswith}\PY{p}{(}\PY{l+s}{\PYZsq{}}\PY{l+s}{J}\PY{l+s}{\PYZsq{}}\PY{p}{)}\PY{p}{]}
\end{Verbatim}

            
                \vspace{-0.2\baselineskip}
            
        \end{ColorVerbatim}
    


    % Make sure that atleast 4 lines are below the HR
    \needspace{4\baselineskip}

    
        \vspace{6pt}
        \makebox[0.1\linewidth]{\smaller\hfill\tt\color{nbframe-in-prompt}In\hspace{4pt}{[}41{]}:\hspace{4pt}}\\*
        \vspace{-2.65\baselineskip}
        \begin{ColorVerbatim}
            \vspace{-0.7\baselineskip}
            \begin{Verbatim}[commandchars=\\\{\}]
\PY{k}{print}\PY{p}{(}\PY{n}{nombres\PYZus{}conj}\PY{p}{)}
\end{Verbatim}

            
                \vspace{-0.2\baselineskip}
            
        \end{ColorVerbatim}
    

    

        % If the first block is an image, minipage the image.  Else
        % request a certain amount of space for the input text.
        \needspace{4\baselineskip}
        
        

            % Add document contents.
            
                \begin{InvisibleVerbatim}
                \vspace{-0.5\baselineskip}
\begin{alltt}['Juan', 'Jaime', 'Jimena']
\end{alltt}

            \end{InvisibleVerbatim}
            
        
    
\subsection{Diccionarios}\label{diccionarios}

Otra de las estructuras de datos nativas en Python son los diccionarios.
Estos, a diferencia de las listas, no es importante el orden de las
entradas y lo único que importa es la asociación \textbf{llave - valor}.
Esto es muy útil cuando nuestros datos tienen ciertas propiedades
descriptivas, como en una agenda telefónica

    % Make sure that atleast 4 lines are below the HR
    \needspace{4\baselineskip}

    
        \vspace{6pt}
        \makebox[0.1\linewidth]{\smaller\hfill\tt\color{nbframe-in-prompt}In\hspace{4pt}{[}42{]}:\hspace{4pt}}\\*
        \vspace{-2.65\baselineskip}
        \begin{ColorVerbatim}
            \vspace{-0.7\baselineskip}
            \begin{Verbatim}[commandchars=\\\{\}]
\PY{n}{agenda} \PY{o}{=} \PY{p}{\PYZob{}}\PY{l+s}{\PYZsq{}}\PY{l+s}{Juan}\PY{l+s}{\PYZsq{}}\PY{p}{:} \PY{l+s}{\PYZsq{}}\PY{l+s}{3234325322}\PY{l+s}{\PYZsq{}}\PY{p}{,} \PY{l+s}{\PYZsq{}}\PY{l+s}{María}\PY{l+s}{\PYZsq{}}\PY{p}{:} \PY{l+s}{\PYZsq{}}\PY{l+s}{7679875568}\PY{l+s}{\PYZsq{}}\PY{p}{,} \PY{l+s}{\PYZsq{}}\PY{l+s}{Pablo}\PY{l+s}{\PYZsq{}}\PY{p}{:} \PY{l+s}{\PYZsq{}}\PY{l+s}{2787316992}\PY{l+s}{\PYZsq{}}\PY{p}{\PYZcb{}}
\end{Verbatim}

            
                \vspace{-0.2\baselineskip}
            
        \end{ColorVerbatim}
    


    % Make sure that atleast 4 lines are below the HR
    \needspace{4\baselineskip}

    
        \vspace{6pt}
        \makebox[0.1\linewidth]{\smaller\hfill\tt\color{nbframe-in-prompt}In\hspace{4pt}{[}43{]}:\hspace{4pt}}\\*
        \vspace{-2.65\baselineskip}
        \begin{ColorVerbatim}
            \vspace{-0.7\baselineskip}
            \begin{Verbatim}[commandchars=\\\{\}]
\PY{k}{print}\PY{p}{(}\PY{n}{agenda}\PY{p}{[}\PY{l+s}{\PYZsq{}}\PY{l+s}{Juan}\PY{l+s}{\PYZsq{}}\PY{p}{]}\PY{p}{)}
\end{Verbatim}

            
                \vspace{-0.2\baselineskip}
            
        \end{ColorVerbatim}
    

    

        % If the first block is an image, minipage the image.  Else
        % request a certain amount of space for the input text.
        \needspace{4\baselineskip}
        
        

            % Add document contents.
            
                \begin{InvisibleVerbatim}
                \vspace{-0.5\baselineskip}
\begin{alltt}3234325322
\end{alltt}

            \end{InvisibleVerbatim}
            
        
    
Otra manera más \emph{Python} de conseguir los valores de una llave es
usar el método \texttt{get}.

    % Make sure that atleast 4 lines are below the HR
    \needspace{4\baselineskip}

    
        \vspace{6pt}
        \makebox[0.1\linewidth]{\smaller\hfill\tt\color{nbframe-in-prompt}In\hspace{4pt}{[}44{]}:\hspace{4pt}}\\*
        \vspace{-2.65\baselineskip}
        \begin{ColorVerbatim}
            \vspace{-0.7\baselineskip}
            \begin{Verbatim}[commandchars=\\\{\}]
\PY{n}{help}\PY{p}{(}\PY{n+nb}{dict}\PY{o}{.}\PY{n}{get}\PY{p}{)}
\end{Verbatim}

            
                \vspace{-0.2\baselineskip}
            
        \end{ColorVerbatim}
    

    

        % If the first block is an image, minipage the image.  Else
        % request a certain amount of space for the input text.
        \needspace{4\baselineskip}
        
        

            % Add document contents.
            
                \begin{InvisibleVerbatim}
                \vspace{-0.5\baselineskip}
\begin{alltt}Help on method\_descriptor:

get(\ldots)
    D.get(k[,d]) -> D[k] if k in D, else d.  d defaults to None.

\end{alltt}

            \end{InvisibleVerbatim}
            
        
    


    % Make sure that atleast 4 lines are below the HR
    \needspace{4\baselineskip}

    
        \vspace{6pt}
        \makebox[0.1\linewidth]{\smaller\hfill\tt\color{nbframe-in-prompt}In\hspace{4pt}{[}45{]}:\hspace{4pt}}\\*
        \vspace{-2.65\baselineskip}
        \begin{ColorVerbatim}
            \vspace{-0.7\baselineskip}
            \begin{Verbatim}[commandchars=\\\{\}]
\PY{k}{print}\PY{p}{(}\PY{n}{agenda}\PY{o}{.}\PY{n}{get}\PY{p}{(}\PY{l+s}{\PYZsq{}}\PY{l+s}{Juan}\PY{l+s}{\PYZsq{}}\PY{p}{,} \PY{l+s}{\PYZsq{}}\PY{l+s}{No existe el contacto}\PY{l+s}{\PYZsq{}}\PY{p}{)}\PY{p}{)}
\end{Verbatim}

            
                \vspace{-0.2\baselineskip}
            
        \end{ColorVerbatim}
    

    

        % If the first block is an image, minipage the image.  Else
        % request a certain amount of space for the input text.
        \needspace{4\baselineskip}
        
        

            % Add document contents.
            
                \begin{InvisibleVerbatim}
                \vspace{-0.5\baselineskip}
\begin{alltt}3234325322
\end{alltt}

            \end{InvisibleVerbatim}
            
        
    


    % Make sure that atleast 4 lines are below the HR
    \needspace{4\baselineskip}

    
        \vspace{6pt}
        \makebox[0.1\linewidth]{\smaller\hfill\tt\color{nbframe-in-prompt}In\hspace{4pt}{[}46{]}:\hspace{4pt}}\\*
        \vspace{-2.65\baselineskip}
        \begin{ColorVerbatim}
            \vspace{-0.7\baselineskip}
            \begin{Verbatim}[commandchars=\\\{\}]
\PY{k}{print}\PY{p}{(}\PY{n}{agenda}\PY{o}{.}\PY{n}{get}\PY{p}{(}\PY{l+s}{\PYZsq{}}\PY{l+s}{Alonso}\PY{l+s}{\PYZsq{}}\PY{p}{,} \PY{l+s}{\PYZsq{}}\PY{l+s}{No existe el contacto}\PY{l+s}{\PYZsq{}}\PY{p}{)}\PY{p}{)}
\end{Verbatim}

            
                \vspace{-0.2\baselineskip}
            
        \end{ColorVerbatim}
    

    

        % If the first block is an image, minipage the image.  Else
        % request a certain amount of space for the input text.
        \needspace{4\baselineskip}
        
        

            % Add document contents.
            
                \begin{InvisibleVerbatim}
                \vspace{-0.5\baselineskip}
\begin{alltt}No existe el contacto
\end{alltt}

            \end{InvisibleVerbatim}
            
        
    
Para saber que llaves contiene un diccionario podemos usar el método
\texttt{keys}, y por ejemplo el método \texttt{items} regresa una lista
de 2-tuplas de la forma \textbf{(llave, valor)}.

    % Make sure that atleast 4 lines are below the HR
    \needspace{4\baselineskip}

    
        \vspace{6pt}
        \makebox[0.1\linewidth]{\smaller\hfill\tt\color{nbframe-in-prompt}In\hspace{4pt}{[}47{]}:\hspace{4pt}}\\*
        \vspace{-2.65\baselineskip}
        \begin{ColorVerbatim}
            \vspace{-0.7\baselineskip}
            \begin{Verbatim}[commandchars=\\\{\}]
\PY{n}{agenda}\PY{o}{.}\PY{n}{items}\PY{p}{(}\PY{p}{)}
\end{Verbatim}

            
                \vspace{-0.2\baselineskip}
            
        \end{ColorVerbatim}
    

    

        % If the first block is an image, minipage the image.  Else
        % request a certain amount of space for the input text.
        \needspace{4\baselineskip}
        
        

            % Add document contents.
            
                \makebox[0.1\linewidth]{\smaller\hfill\tt\color{nbframe-out-prompt}Out\hspace{4pt}{[}47{]}:\hspace{4pt}}\\*
                \vspace{-2.55\baselineskip}\begin{InvisibleVerbatim}
                \vspace{-0.5\baselineskip}
\begin{alltt}dict\_items([('Juan', '3234325322'), ('María', '7679875568'), ('Pablo',
'2787316992')])\end{alltt}

            \end{InvisibleVerbatim}
            
        
    


    % Make sure that atleast 4 lines are below the HR
    \needspace{4\baselineskip}

    
        \vspace{6pt}
        \makebox[0.1\linewidth]{\smaller\hfill\tt\color{nbframe-in-prompt}In\hspace{4pt}{[}48{]}:\hspace{4pt}}\\*
        \vspace{-2.65\baselineskip}
        \begin{ColorVerbatim}
            \vspace{-0.7\baselineskip}
            \begin{Verbatim}[commandchars=\\\{\}]
\PY{k}{for} \PY{n}{nombre}\PY{p}{,} \PY{n}{telefono} \PY{o+ow}{in} \PY{n}{agenda}\PY{o}{.}\PY{n}{items}\PY{p}{(}\PY{p}{)}\PY{p}{:}
    \PY{k}{print}\PY{p}{(}\PY{l+s}{\PYZsq{}}\PY{l+s}{El número de \PYZob{}0\PYZcb{} es \PYZob{}1\PYZcb{}}\PY{l+s}{\PYZsq{}}\PY{o}{.}\PY{n}{format}\PY{p}{(}\PY{n}{nombre}\PY{p}{,} \PY{n}{telefono}\PY{p}{)}\PY{p}{)}
\end{Verbatim}

            
                \vspace{-0.2\baselineskip}
            
        \end{ColorVerbatim}
    

    

        % If the first block is an image, minipage the image.  Else
        % request a certain amount of space for the input text.
        \needspace{4\baselineskip}
        
        

            % Add document contents.
            
                \begin{InvisibleVerbatim}
                \vspace{-0.5\baselineskip}
\begin{alltt}El número de Juan es 3234325322
El número de María es 7679875568
El número de Pablo es 2787316992
\end{alltt}

            \end{InvisibleVerbatim}
            
        
    
Para actualizar algún elemento del diccionario usamos el método
\texttt{update}:

    % Make sure that atleast 4 lines are below the HR
    \needspace{4\baselineskip}

    
        \vspace{6pt}
        \makebox[0.1\linewidth]{\smaller\hfill\tt\color{nbframe-in-prompt}In\hspace{4pt}{[}49{]}:\hspace{4pt}}\\*
        \vspace{-2.65\baselineskip}
        \begin{ColorVerbatim}
            \vspace{-0.7\baselineskip}
            \begin{Verbatim}[commandchars=\\\{\}]
\PY{n}{agenda}\PY{o}{.}\PY{n}{update}\PY{p}{(}\PY{p}{\PYZob{}}\PY{l+s}{\PYZsq{}}\PY{l+s}{Juan}\PY{l+s}{\PYZsq{}}\PY{p}{:} \PY{l+s}{\PYZsq{}}\PY{l+s}{324453532}\PY{l+s}{\PYZsq{}}\PY{p}{\PYZcb{}}\PY{p}{)}
\end{Verbatim}

            
                \vspace{-0.2\baselineskip}
            
        \end{ColorVerbatim}
    


    % Make sure that atleast 4 lines are below the HR
    \needspace{4\baselineskip}

    
        \vspace{6pt}
        \makebox[0.1\linewidth]{\smaller\hfill\tt\color{nbframe-in-prompt}In\hspace{4pt}{[}50{]}:\hspace{4pt}}\\*
        \vspace{-2.65\baselineskip}
        \begin{ColorVerbatim}
            \vspace{-0.7\baselineskip}
            \begin{Verbatim}[commandchars=\\\{\}]
\PY{k}{for} \PY{n}{nombre}\PY{p}{,} \PY{n}{telefono} \PY{o+ow}{in} \PY{n}{agenda}\PY{o}{.}\PY{n}{items}\PY{p}{(}\PY{p}{)}\PY{p}{:}
    \PY{k}{print}\PY{p}{(}\PY{l+s}{\PYZsq{}}\PY{l+s}{El número de \PYZob{}0\PYZcb{} es \PYZob{}1\PYZcb{}}\PY{l+s}{\PYZsq{}}\PY{o}{.}\PY{n}{format}\PY{p}{(}\PY{n}{nombre}\PY{p}{,} \PY{n}{telefono}\PY{p}{)}\PY{p}{)}
\end{Verbatim}

            
                \vspace{-0.2\baselineskip}
            
        \end{ColorVerbatim}
    

    

        % If the first block is an image, minipage the image.  Else
        % request a certain amount of space for the input text.
        \needspace{4\baselineskip}
        
        

            % Add document contents.
            
                \begin{InvisibleVerbatim}
                \vspace{-0.5\baselineskip}
\begin{alltt}El número de Juan es 324453532
El número de María es 7679875568
El número de Pablo es 2787316992
\end{alltt}

            \end{InvisibleVerbatim}
            
        
    

        

        \renewcommand{\indexname}{Index}
        \printindex

    % End of document
    \end{document}


